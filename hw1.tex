\documentclass[12pt]{article}
\usepackage[margin=.5in]{geometry}
\title{\vspace*{-3em}HW1}
\author{William Hua}
\date{\today}
\begin{document}
\maketitle

\section*{1}
\begin{description}
\item[a] Pain free in the treatment group is $\frac{10}{43}$. Pain free in the 
    control group is $\frac{2}{46}$
\item[b] At first glance it does seem that acupuncture is an effective treatment
    for migraines because the treatment group has such a higher percentage of 
    patients who report being pain free after 24 hours than the control/placebo
    group.
\item[c] It could be due to chance, but there is pretty strong evidence to 
    suggest otherwise. One would need to find a p-value to find the exact
    probability of error.
\end{description}

\section*{2}
\begin{description}
\item[a]
\item[b]
\item[c]
\end{description}

\section*{3}
\begin{description}
\item[a]
\item[b]
\item[c]
\end{description}

\section*{4}
\begin{description}
\item[a]
\item[b]
\item[c]
\end{description}

\section*{7}
\begin{description}
\item[a]
\item[b]
\item[c]
\end{description}

\section*{16}
\section*{17}
\section*{22}

\section*{1.23}
%FIXME: load plot into R, fit a line through it
\begin{description}
\item[a] Positive correlation
\item[b] Same, positive correlation. The reason I believe so is because if X
    increase when Y increases, than Y should also increase when X increases
\item[c] I do not believe life span and gestation are independent because there
    is a pretty clear positive association between the variables (even though
    it has high variance).
\end{description}

\section*{1.29}
%FIXME: reproduce the histogram
%do these people smoke more on weekends than during the week?
The mean for smoking on weekends is a little bit higher than smoking on weekdays
(16.4 vs 13.7). Both are heavily skewed upwards with possible outliers above 50. %FIXME: possible outliers?
They both seem to have similar standard deviations with weekends ending up a 
little higher (9.8 vs 9.3).
\end{document}
