\documentclass{article}
\usepackage{longtable}
\usepackage[margin=.5in]{geometry}
\title{\vspace*{-3em}HW7}
\author{Wil Hua}
\date{\today}
\begin{document}
\maketitle
% Chapter 6, problems 1, 5,8,10, 15, 16, 18, 27, 29, 32
\section*{6.1}
\begin{description}
    \item[a)] False. Success-failure condition not met. 
    \item[b)] True. It is not normally distributed due to the failure to meet 
        the success-failure condition. It is right skewed because it is bounded
        by 0, meaning that far left skew probabilities are not possible and will
        bunch up more than in the opposite direction.
    \item[c)] False. Z score is 1.65 which is not enough to be considered an outlier.
    \item[d)] True. Z score is 2.32 which is enough to be considered an outlier.
    \item[e)] False. Decreases SE by square root of 1/2 since it's standard error
        and not variance. It scales by roots. 
\end{description} 


\section*{6.5}
\begin{description}
    \item[a)] False. We are 100\% certain that 70\% of voters in THIS SAMPLE
        will support Prop 19 since we know the true proportion for this sample.
    \item[b)] True, unless this is year much different from 2010 (ie 1850).
    \item[c)] True for the year 2010
    \item[d)] True, while maintaining everything else as equal
    \item[e)] True for the year 2010 since the mean proportion is above 50\%
        by a statistically significant amount.
\end{description} 

\section*{6.8}
\begin{description}
    \item[a)] $\sqrt{\frac{(.66*.34)}{1018}} * 1.96 = .029 ~ .03$
    \item[b)] No, the confidence intervals at 95\% are 63 and 69, meaning
        that greater than 69 (ie 70) indicates the cutoff region where 
        there is no longer 95\% confidence that the proportion is true.
\end{description}

\section*{6.10}
\begin{description}
    \item[a)] The percentage of the citizens that categorized their poorly
        enough to be considered "suffering". (.25)
    \item[b)] This is a random sample from less than 10\% of the population
        so the observations can be assumed to be independent. The success-failure
        condition is also met since 250 and 750 are both well above 10.
    \item[c)] $\sqrt{\frac{.25*.75}{1000}} * 1.96 +\- .25 = (.2231, .2768)$
    \item[d)] If we were to use a higher confidence interval the interval would 
        expand since it would require a higher z score to multiply the standard 
        error.
    \item[e)] If we were to use a larger sample the interval would shrink since
        it would reduce the standard error by a factor of $\frac{1}{\sqrt{n}}$
\end{description}

\section*{6.15}
\begin{description}
    \item[a)] $H_0: p = .38, H_A: p \neq .38$ The sample is less than 10\% of 
        the population and the success failure conditions are satisfied. Z score
        is VERY large and p-value is almost 0, meaning that there is almost no chance
        the null hypothesis is true. This implies that the proportion of US citizens
        that use their mobile phones to browse the internet is significantly below
        the proportion of Chinese citizens that use their mobile phones to browse
        the internet.
    \item[b)] In this context (for a two tailed test) it represents the probability
        that the proportion of Chinese citizens that use their mobile phones to browse
        the internet has almost no chance of being the same as the proportion of 
        US citizens that use their mobile phones to browse the internet.
    \item[c)] $\sqrt{\frac{.17*.83}{2254}} * 1.96 +/- .17 = (.1544, .1855)$
\end{description} 

\section*{6.16}
\begin{description}
    \item[a)] $H_0: p = .5, H_A: p < .5$ The sample is less than 10\% of the 
        population and the success failure conditions are satisfied. Standard error
        is $\sqrt{\frac{.48*.52}{331}} = .0274$. Z score is .72. This means that
        we do not reject the null hypothesis, or that there is not enough data 
        to support the claim that the minority of Americans who chose not to go 
        to college did so due to financial reasons.
    \item[b)] It should contain .5 because we were not confident that .5 is 
        signficantly than the sample proportion.
\end{description}
\section*{6.18}
\begin{description}
    \item[a)] The sample size is less than 10\% of the population size, and 
        the failures/successes are both above 10. The 90\% confidence 
        interval:$.0274 * 1.64 +/- .48 = (.4350, .5249)$. We are 90\% confident
        that between .435 and .5249 of all students who did not go to university
        did so due to financial reasons
    \item[b)] $\sqrt{\frac{.48*.52}{n}} = .015/1.64$, $n = 2984$
\end{description}

\section*{6.27}
\begin{description}
    \item[a)] $\sqrt{\frac{.7*.3}{819} + \frac{.42*.58}{783}} * 1.96 +/- .7 - .
        42 = (.23, .33)$. We are 95\% confident that the porportion of Democrats
        who support the health care public option plan in 2009 is 23\% to 33\%
        higher than the proportion of Independents.
    \item[b)] True, it is more likely for the Democrat to support it than the
        Independent
\end{description}

\section*{6.29}
\begin{description}
    \item[a)] College graduates = $\frac{104}{438} = .2374$. Non college grad = 
        $\frac{131}{389} = .3367$.
    \item[b)] $H_0: CG = NCG, H_A: CG \neq NCG$. It is most likely that 389 is 
        less than 10\% of the non college grad population and 438 is also less
        than the 10\% of the college grad population. Both proportions also
        satisfy the failure-success criterion. Standard error is $\sqrt{
        \frac{.2374*(1-.2374)}{438} + \frac{.3367*(1-.3367)}{389}}$. 
        Z score is $-3.18$ and p-value is $.0014$ so we reject the null 
        hypothesis. The data supports the conclusion that proportion of 
        college graduates that are of the opinion that they know enough to 
        have an educated stance on the issue is greater than the 
        proportion of non college graduates (p-value is small enough
        to imply directionality in a one tailed test).
\end{description}

\section*{6.32}
\begin{description}
    \item[a)] All assumptions are met. $H_0: R = D, H_A R \neq D$. 
        Standard error is $\sqrt{\frac{.83*.17}{318} + \frac{.81*.19}{369}} = 
        .0293$. $\frac{.83 - .81}{.0293} = .68$. P-value is $.24$. Do not 
        reject the null hypothesis. The data does not support the conclusion
        that the proportions are different.
    \item[b)] Type II error because we would be making the error
        that despite the null hypothesis being false, we failed to reject it.
\end{description}

\end{document}
